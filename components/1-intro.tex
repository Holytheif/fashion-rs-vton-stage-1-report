\chapter{INTRODUCTION}

\section{Details of project work}
	In the dynamic world of e-commerce, the advent of online clothing shopping has ushered in a new era of convenience, offering consumers an expansive array of fashion choices at their fingertips. This digital transformation has enriched the shopping experience, allowing people to explore styles and trends from the comfort of their own homes. However, amidst the convenience and variety, a significant challenge looms large: the absence of the traditional "try-before-buy" component. The inability to physically try on clothing before making a purchase decision introduces a layer of uncertainty, which can be a daunting prospect for consumers. This uncertainty often leads to hesitant buying decisions, elevated return rates, and ultimately, dissatisfaction among online shoppers.

	The "AI-Based Clothing Recommendation for Try Before Buy" project takes centre stage to confront this pressing issue head-on. By combining the capabilities of Artificial Intelligence (AI) and Augmented Reality (AR), our mission is to create a seamless junction between the digital and physical realms, empowering consumers to make confident and informed clothing choices while shopping online. This innovative approach is poised to revolutionise the online clothing shopping experience.

	Online clothing shopping presents multifaceted challenges for consumers. The inability to physically touch, try on, and assess garments before making a purchase decision generates an array of uncertainties. Shoppers grapple with perpetual uncertainty, unsure whether their selected item will meet their style and fit expectations, leading to hesitancy in making online clothing purchases. This uncertainty, coupled with the variance in sizing and style preferences across brands, often results in high return rates in the e-commerce sector, which significantly impact both customers and retailers. Consumers frequently order multiple sizes or styles of the same item to secure the best fit, leading to additional expenses and environmental waste through unnecessary returns. Furthermore, the absence of a genuine try-on experience can erode overall customer satisfaction, potentially undermining brand loyalty.

	The "AI-Based Clothing Recommendation for Try Before Buy" project is positioned to mitigate these challenges by introducing an advanced recommendation system, surpassing traditional suggestions by understanding individual preferences and providing a sophisticated Augmented Reality (AR) try-on experience. This innovative approach empowers users to visualise and evaluate how clothing items align with their style, size, and fit in real-time, thereby bridging the gap between online and in-store shopping experiences.
