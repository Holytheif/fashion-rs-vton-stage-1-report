\chapter[Introduction]{INTRODUCTION}

\section{Details of project work}
	This project comprises several interconnected components, each contributing to a seamless and immersive experience. Our work is has two main goals. One is to focus on the development of a recommendation system using collaborative and content-based filtering algorithms to provide personalized clothing recommendations by considering user preferences, past history, clothing item compatibility, and outfit possibilities. And the other is to develop an image synthesis-integrated virtual try-on system to offer real-time interactivity with clothing items. This system will utilize advanced computer vision and image processing techniques to generate realistic representations of users wearing selected clothing items. Throughout the project, we have to keep user experience in mind and develop a robust feedback mechanism to continually improve the generated results.

\section{Objective}
	The objectives of this project are as follows:

	\begin{enumerate}
		\item \textbf{Personalized clothing recommendation:} This objective aims to provide users with highly personalized clothing recommendations based on their individual preferences and needs. For size, the system takes into account the user's body measurements and style preferences. For price, it considers the user's budget constraints. Regarding color, it suggests clothing items that align with the user's preferred color palette.
		\item \textbf{Enhanced User Engagement Strategies:} Increasing user engagement involves making the online shopping experience more interactive, enjoyable, and informative. The AI-based clothing recommendation and try-on system aims to engage users by allowing them to virtually try on clothing, providing realistic visuals of how items fit and look.
		\item \textbf{Increased Conversion Rates:} Conversion rate is a critical metric for e-commerce companies. The project aims to increase conversion rates by reducing the friction in the buying process. When users can virtually try on clothing and receive tailored recommendations, they are more likely to make purchase decisions.
		\item \textbf{Reduction of Return Rates:} High return rates are a significant challenge in e-commerce, often due to issues with sizing, style, or fit. The project aims to tackle this problem by allowing users to virtually try on clothing items, helping them make more informed purchase decisions.
		\item \textbf{Increased Revenue for Businesses:} Increasing revenue is a primary objective for e-commerce companies. The project aims to achieve this by offering personalized recommendations and an immersive try-on experience. Users who feel confident in their clothing choices are more likely to complete purchases.
		\item \textbf{Data Privacy and Security Compliance:} Data privacy and security are paramount in this project. Protecting user data, including personal information and image synthesis try-on images, is a core objective. Robust data security measures, such as encryption, access controls, and compliance with data protection regulations, are in place to ensure user trust and privacy. Our commitment to safeguarding user data extends throughout the development and deployment phases of the image synthesis-integrated virtual try-on system.
		\item \textbf{Integration with AI assistants:} The project aims to lay groundwork for future exploration of opportunities for seamless integration with AI assistants, such as voice-activated and chat-based virtual shopping advisors. This integration can enhance user interactions, provide real-time style advice, and further streamline the online shopping process.
	\end{enumerate}

\section{Scope}
	The scope of this project is as follows:

	\begin{enumerate}
		\item \textbf{Fashion embedding:} Fashion embedding is a pivotal component, serving as the foundation for further tasks.
		\item \textbf{Feature extractor:} Use deep learning to train a feature extractor which can segment parts of the upper body to provide user features for attribute-based recommendations, and to detect rendering regions for virtual try-on.
		\item \textbf{AI-based recommender:} Design and deploy a recommendation system that utilizes collaborative and content-based filtering to generate personalized clothing recommendations.
		\item \textbf{Image-synthesis virtual try-on:} Develop an image synthesis-based virtual try-on system that allows users to visualize and interact with clothing items in real-time.
	\end{enumerate}

\section{Motivation}
	Our motivations for this project are as follows:

	\begin{enumerate}
		\item \textbf{Uncertainty in decisions:} Online shoppers face an overwhelming array of clothing choices, often leading to hesitation and indecision. The abundance of options can make it challenging for users to make confident purchasing decisions, resulting in abandoned shopping carts and high return rates.
		\item \textbf{Poor online shopping experience:} The inability to try on garments before purchase has long been a pain point for consumers. Many potential buyers grapple with concerns about how garments will look on their bodies. This uncertainty can hinder their ability to make informed purchasing decisions and ultimately affect their overall satisfaction with the shopping process.
		\item \textbf{Environmental impact:} The issue of high return rates in e-commerce, particularly in the fashion sector, has far-reaching environmental consequences. Consumers often resort to ordering multiple sizes or styles of the same item, leading to excess packaging waste and unnecessary transportation emissions associated with product returns.
		\item \textbf{Technological innovation:} The motivations for this project are aligned with the pursuit of technological innovation. We recognize that AI and image synthesis are two of the most exciting and transformative technologies of our time. Our motivation is to harness the power of these technologies and demonstrate their real-world applications in the e-commerce fashion sector. By leveraging AI and image synthesis, we aim to revolutionize the way people shop for clothing online, providing them with a more immersive and personalized experience. This project serves as a testament to the possibilities offered by cutting-edge technologies in enhancing various aspects of our daily lives.
		\item \textbf{Customer-centric approach:} A strong motivation for this project is a commitment to a customer-centric approach. We believe that every technological advancement and innovation should ultimately serve the needs and desires of consumers. The motivation is to create a solution that places users at the centre of the online shopping experience, giving them the tools and information they need to make choices that reflect their individuality and preferences.
	\end{enumerate}